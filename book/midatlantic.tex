\chapter{Mid-Atlantic Problem Sets}

This chapter contains some notes about the problems occurring in the Mid-Atlantic
problem set.  We focus on this corpus in particular because there are recurring
themes since the problems have been created by the same person (or team) for
multiple years.

\section{2005}

\href{http://midatl.radford.edu/docs/pastProblems/05contest/MidAtlantic2005.pdf}{(Problem Set PDF 2005)}

\subsection{C Extrusion}
\label{sec:2005-c-extrusion}

Straightforward application of polygon area formula, see Section~\ref{sec:areapolygon}.

\section{2006}
\href{http://midatl.radford.edu/docs/pastProblems/06contest/MidAtlantic2006.pdf}{(Problem Set PDF 2006)}

\subsection{E Marbles}
\label{sec:2006-e-marbles}
A simple state space exploration problem solvable with straightforward BFS exploration.
Catch: judge input data missed the "0 0 0" line.

\section{2007}
\href{http://midatl.radford.edu/docs/pastProblems/07contest/MidAtlantic2007.pdf}{(Problem Set PDF 2007)}

\subsection{B Mobiles Alabama}
\label{sec:2007-b-mobile}

Lexical analysis benefits from zero-width lookaround~\ref{sec:lookaroundsplitting}, although
simpler solutions may work, too, such as replacing '(' and ')' with ' ( ' and ' ) ' before 
splitting on whitespace.
Recursive descent parsing should be used to analyze the syntactical structure of the input.

\subsection{C Out Of Sight}
\label{sec:2007-c-sight}

Dynamic programming.  

Applying BFS will time out due to an explosion in the number
of successor states, most of which will have been already seen.

\subsection{D Witness Redaction}
\label{sec:2007-d-witness}
This problem can be solved with regular expressions and zero-width lookaround splitting.
See Section~\ref{sec:lookaroundsplitting}.

\section{2008}
\href{http://midatl.radford.edu/docs/pastProblems/08contest/MidAtlantic2008.pdf}{(Problem Set PDF 2008)}

\subsection{G Stems Sell}
\label{sec:2008-g-stems}

Can be solved with regular expressions.

\href{http://midatl.radford.edu/docs/pastProblems/08contest/JudgingData/G-stems/}{Judge data}
appears broken, even on ICPC site.

%%%%%%%%%%%%%%%%%%%%%%%%%%%%%%%%%%%%%%%%%%%%%%%%%%%%%%%%%%%%%%%%%%%%%%%%%%%%%%%%%%%%%%%%%%%%%%%%%%%%%

\section{2009}
\href{http://midatl.radford.edu/docs/pastProblems/09contest/MidAtlantic2009.pdf}{(Problem Set PDF 2009)}

\subsection{G Stringer}
\label{sec:2009-g-stringer}

Combinatorics and recursion.  The instructions says that K fits into a 32-bit integer, but 
the judge data in fact contains K that require 64-bit integers.  I'm guessing this was clarified
during the contest.

A brute force approach (e.g., STL \code{next\_permutation} or manual generation of permutations) 
will not work.

%%%%%%%%%%%%%%%%%%%%%%%%%%%%%%%%%%%%%%%%%%%%%%%%%%%%%%%%%%%%%%%%%%%%%%%%%%%%%%%%%%%%%%%%%%%%%%%%%%%%%

\section{2010}
\href{http://midatl.radford.edu/docs/pastProblems/10contest/MidAtlantic2010.pdf}{(Problem Set PDF 2010)}

\subsection{C Cells}
\label{sec:2010-c-cells}

Solvable using Monte Carlo simulation.  Can use java.awt.geom.* to implement containment check.
Also solvable using numerical integration (in polar coordinates).

\subsection{D Not One Bit More}
\label{sec:2010-d-bit}

Requires combinatorics and recursion.
Note that the provided range ($HI <= 10^{18}$) rules out any brute force approach.

%%%%%%%%%%%%%%%%%%%%%%%%%%%%%%%%%%%%%%%%%%%%%%%%%%%%%%%%%%%%%%%%%%%%%%%%%%%%%%%%%%%%%%%%%%%%%%%%%%%%%

\section{2011}

\href{http://midatl.radford.edu/docs/pastProblems/11contest/MidAtlantic2011.pdf}{(Problem Set PDF 2011)}

\subsection{B Raggedy, Raggedy}
\label{sec:2011-b-raggedy}

This problem can be solved using dynamic programming.
Note that there may be leading spaces on some input lines.

\subsection{E Losers are Winners}
\label{sec:2011-e-losers}

An easy sorting problem.
During the contest, many students overlooked the case that a team's weight could go up.
Beware of ``clever'' solutions that fail for the $n=1$ case, see~\ref{sec:gotchas}.

\subsection{F Line of Sight}
\label{sec:2011-f-lineofsight}

Straightforward application of area of polygon~\ref{sec:areapolygon} and line intersection~\ref{sec:lineintersection}.
Note that parameters of the problem even exclude corner cases for line intersection (e.g. parallel lines, right angles).

\subsection{H Road Rally}
\label{sec:2011-h-rally}

Straightforward state space exploration using BFS, see Section~\ref{sec:bfs}.

Input handling requires care since handout is misleading by showing race tracks enclosed in 'x' which is not
required.  Also, the 'horizontal' and 'vertical' distance is to be treated signed, not absolute.

