\documentclass[10pt]{book} 

\usepackage{amsmath}
\usepackage{palatino}
\usepackage{parskip}
\usepackage{minted}
\usepackage[margin=1in,letterpaper]{geometry}
\usepackage{fancyvrb}
\usepackage{color}
\definecolor{maroon}{RGB}{66,00,00}

\usepackage{hyperref}
% http://en.wikibooks.org/wiki/LaTeX/Hyperlinks
\hypersetup{
    bookmarks=true,         % show bookmarks bar?
    unicode=false,          % non-Latin characters in Acrobat’s bookmarks
    pdftoolbar=true,        % show Acrobat’s toolbar?
    pdfmenubar=true,        % show Acrobat’s menu?
    pdffitwindow=false,     % window fit to page when opened
    pdfstartview={FitH},    % fits the width of the page to the window
    pdfnewwindow=true,      % links in new window
    colorlinks=true,       % false: boxed links; true: colored links
    linkcolor=maroon,          % color of internal links
    citecolor=green,        % color of links to bibliography
    filecolor=magenta,      % color of file links
    urlcolor=maroon         % color of external links
}
\usepackage{url}
\usepackage{fancyhdr}
% \pagestyle{fancy}
\usepackage{makeidx} %If you want to generate an index, automatically 
\usepackage{graphicx} %If you want to include postscript graphics 
\makeindex
%
% Some definitions
%
\def\code#1{\texttt{#1}}
\def\inputjava#1{\inputminted[fontsize=\footnotesize,linenos=true]{java}{code/#1.java}}


\begin{document} 

\author{ACM Programming Team 2012} 
\title{VT ACM ICPC Handbook} 
\date{\today{}} 

%
% This file defines commands for each figure.
% defining a command allows us to more easily change 
% where the figure will go.
%
\graphicspath{{images/}}

%\newcommand{\linelineintersectionfigure}{
%    \begin{figure}
%        \centering
%        % Wikipedia Public Domain image
%        \includegraphics[height=2in]{Line-Line_Intersection.png}
%        \caption{Line line intersection}
%        \label{fig:linelineintersect}
%    \end{figure}
%}


 

\frontmatter 
\maketitle
\tableofcontents 
\chapter{Preface}

This book is intended as a reference, to be used both during the competition as well
in preparation for it.

It is hosted on github at
\href{https://github.com/VTACMProgrammingTeam/ICPCHandbook}{https://github.com/VTACMProgrammingTeam/ICPCHandbook}.
If you wish to contribute, please send email to godmar@gmail.com

 

\mainmatter 
\chapter{Standard Libraries}

The ACM ICPC, at the time of this writing, allows the use of the JDK 1.6 libraries as well
as the C++ STL.  Knowledge and mastery of the existing JDK classes is crucial for success.
This chapter reviews some of the more commonly used classes.

\section{Collection Classes}

\subsection{Object Equivalence}
\label{sec:objequivalence}

Java allows you to define and equivalence relationship between objects using the
\code{Object.equals} method, which is required for some problems.  If implemented,
a conforming \code{hashCode} method must be implemented as well.
Here are some hints.

\begin{itemize}
\item \textbf{Avoid gratuitous implementations.}  In the vast majority of cases,
you will not need your own \code{equals/hashCode} function!  You only need equals/hashCode
if your object is used as a key (not value!) in a Map or Set, \textbf{and} if the
default implementation of equals (``two objects are equal if they are the same'') does
not suffice.   Simply needing to store objects in a set, or using them as a map key,
is not a justification for implementing equals().
An example are search problems in a state space - you may create a 
state instance that is equal to an already explored state.

\item \textbf{Implement equals() correctly}.
You need to compare all relevant fields to one another.
\begin{minted}{java}
@Override
public boolean equals(Object _that) {
    State that = (State)_that;
    return this.field1 == that.field1 && ... && this.fieldn == that.fieldn;
}
\end{minted}
For clarity, I recommend for \code{equals()} to always include all fields, and 
to not include irrelevant fields in the object.  (That is also why I like separate
previous hop maps rather than previous hop fields in BFS implementations,
see Section~\ref{sec:bfs}.)

\item \textbf{Understand the equals/hashCode contract}.
The contract says two objects that are equal must have the same hashCode().
It does not require that if two objects have the same hashCode() they must
be equals().  There will be, in general, multiple objects mapping to the
same hashCode().

\item \textbf{Implement hashCode() efficiently.} 
XXX CONTINUE HERE

\item \textbf{Consider hashCode()'s distribution.} 
The contract would be met by a degenerate function that returns always zero, but this
would create terrible performance - in a hash map, all objects would be mapped
to the same slot, resulting in linear lookup performance.

\end{itemize}

 
\chapter{Simulation}

While many simulation problems can be solved ad-hoc, some require, or may benefit
from, a more principled approach.

\section{Monte Carlo Methods}

In some cases, random sampling might provide a solution.
For instance, the code below computes $\pi$ by examining whether
randomly produced $(x, y)$ samples are inside or outside the unit 
circle in the northeast quadrant.

\inputminted[fontsize=\footnotesize,linenos=true]{java}{code/Pi.java}

\paragraph{Precision} is generally not very good, mainly due to the  
statistical properties of the underlying pseudo-random number generator (PRNG).
Increasing the number of samples will yield diminishing returns quickly.
As a rule of thumb, don't expect more than 4 significant digits.
For example, 2010/Cells~\ref{sec:2010-c-cells} asks for a percentage with 2 digits after the
period, for a total of 4 significant digits, making this approach feasible.

\section{Discrete Event Simulation}

Discrete Event Simulation is widely used to simulate physical phenomena, especially
those that involve periodic and/or random events, as well as when events may or may
not occur, depending on the state of the simulation.

The key idea is to maintain a timeline as a queue of future events, sorted
by their timestamp.  Simulation time, aka virtual time, advances in jumps when
the next upcoming event is pulled off the queue.  Time jumps over period in which
no events are scheduled (unlike a continuous simulation in which time increases
in small, but constant, steps).

Event handlers may query the current time and they may schedule future events.
Discrete Event simulations contain a typical core: a way to represent events, 
a queue to sort them, and an event loop that processes the event queue until
either the maximum simulation time is reached or a solution to the problem is found.
Recommended implementation strategy is to use immutable event objects that are
inserted when scheduled and discarded after dispatch.

Example: \href{http://uva.onlinejudge.org/index.php?option=com_onlinejudge&Itemid=8&category=3&page=show_problem&problem=97}{UVA 00161}
is an example of a problem which can be solved with DES.  Note, however, that the small simulation time
frame (18,000 seconds) also allows a continuous simulation approach, simply simulating every second,
rather than only those points in time when a traffic light changes.
This is a very simple example, with only 3 event types, all implemented in one
class LightChange.

\inputminted[fontsize=\footnotesize,linenos=true]{java}{code/UVA00161TrafficLights.java}

 
\chapter{Searching}
\def\astar{$A^*$}

\section{Backtracking}
\label{sec:backtracking}

Bitner~\cite{Bitner:1975}.

Include n-Queens examples.


 

\section{State Space Exploration}

In these problems, an initial state is to be transformed through a series of valid
moves into one goal state, or possibly one of multiple possible goal states. 
Examples include block puzzles or single-player games.
These problems have the following characteristics:
\begin{itemize}
\item An optimal solution required: we want the minimal number of moves from the
initial state to goal state, rather than just any.

\item No obvious strategy.  Given a state, there's no obvious way to choose which
move to make to get closer to the goal.  In fact, it's usually even difficult to
tell how far we might be away from the goal state.  Any easy-to-find lower bounds
for this distance to the goal might be far too optimistic.  
See also \astar{} in Section~\ref{sec:astar}.

\end{itemize}

\subsection{Breadth-First Search (BFS)}
\label{sec:bfs}
There are numerous ways of writing a BFS loop.  There are even more ways of
getting it wrong.
Here is one possible way:

\inputminted[fontsize=\footnotesize,linenos=true]{java}{code/bfsloop.java}

\paragraph{Notes.}  
\begin{itemize}
\item Adding the final state could be avoided by calling output() if a 
final state is found, potentially saving the unnecessary expansion of states
if the optimal goal state is already in the queue - however, then the case where 
the initial state is final must be handled separately.  
Seen in 2006/E Marbles~\ref{sec:2006-e-marbles} where the judge data contains
the case that the initial state is final.

\item The state class should use ducktyping and provide the necessary
methods - isfinal, successors.  In addition, the state class must implement
an object equivalence relationship (equals, hashCode) as described in 
Section~\ref{sec:objequivalence}.

\item The example keeps track of both the path (via 'pred') and distance (via 'dist').
    If the problem asks for only one of the two, the other can be omitted. In that
    case, the insertion of a successor state should be guarded by the one remaining.
    Note that both 'pred' and 'dist' have the invariant when encountering a state 
    multiple times, only the first encountered state is kept in the map.

\end{itemize}

\paragraph{When Not To Use BFS.}

There are some problems that may appear to be solvable using BFS, but are in fact 
dynamic programming problems.  These are generally problems in which there are
many transitions from states farther in the problem space to states that have
been discovered much earlier.
An example is 2007/C/Out of Sight.~\ref{sec:2007-c-sight}.

\subsection{\astar}
\label{sec:astar}

\astar~\cite{Pearl:1984} is a classic algorithm to improve upon breadth-first search when an
estimator function for states that estimates their distance to the goal state is known.
Though an AI algorithm, it can yield guaranteed optimal solution if the estimator is chosen
carefully.

The idea is simple.  In BFS, we're examining nodes based on their distance from the start state. 
So we'll look at, and expand, all states that are 6 moves away from the start *before* we 
look at any state that's 7 moves from the start - simply because of the FIFO discipline in the queue.

In \astar, instead of putting all to-be-explored states in a FIFO queue,
we sort them by their "goodness," which is defined as a
lower bound of how many moves a solution is away from the state.
For instance, if a state A is 6 moves away from the start, and
it is known that it is at least 10 moves away from the goal,
it would have a goodness of 16 (6 + 10).  On the other hand, if a
move B is 8 moves away from the start, and at least 4 moves away
from the solution, its goodness is 12 and it's explored first.
Note that we are not estimating how far the state is from
the solution. We're simply constructing a function that says: this
state is *at least* this far from a solution. It may be farther -
our decision to explore state B may be wrong and the closest solution
state may result from the exploration of A.

The challenge lies in how to say - quickly - that a state is
``at least this many moves away from the goal.''  In general, that's
tough since if we knew how many states a state is away from the goal,
it would be much easier to solve the problem. However, there are some lower
bounds that can be found - for instance, in a puzzle, we could count how
many pieces are not in their final position, knowing that it'll take at
least this many moves to get them there (assuming it's a puzzle like
the traditional 15-piece puzzle where all pieces need to be sorted).
Another possible function is the sum of the Manhattan
distances between each piece's current and final position.  
Again, it is clear that it will take at least this many moves.

In competition problems you are practically never asked to find an 'almost'
perfect heuristic solution - they ask
you to compute the correct, and in search problems like these, the
shortest solution.  \astar{} will provide you with the optimal
solution if the estimation function does not overestimate  - if it
did, it might lead to the nearest goal state being placed behind a
farther goal state in the queue.  That farther state might then be
discovered first, and mistaken for the shortest solution.

It's not a problem that the estimation function underestimates - in fact,
it'll generally do that - this just means that \astar might waste some
time exploring states it thinks are promising, but which do not actually
lead to a solution.

Converting a BFS to A-Star in Java is really simple.  Replace the entry
of the BFS described in Section~\ref{sec:bfs} such that the BFS work
queue uses a PriorityQueue like so:

\inputminted[fontsize=\footnotesize,linenos=true]{java}{code/astar.java}

Note how making dist final allows it to be used in the comparator.
The code exploits that both PriorityQueue and ArrayDeque implement the
Queue interface.

\paragraph{Notes}  
\begin{itemize}
\item
\astar will increase the cost of inserting and removing a state from the queue
substantially - in a heap-based priority queue, 'offer()' and 'poll()' take O(log n) whereas 
they are constant time O(1) operations on a Deque.  On the other hand, especially if the
estimator function is too optimistic, the state space may only be marginally reduced.
Recommendation: use A-Star only if BFS times out and/or if a good heuristic
can be found.

\end{itemize}
 
\chapter{String Processing}

\section{Regular Expressions}

\subsection{Regular Expressions}

\subsection{Zero-width Lookahead Technique}

\begin{figure}
    \centering
    % Wikipedia Public Domain image
    \includegraphics[height=1.5in]{lookaheadsplitting.png}
    \caption{Lookahead Splitting}
    \label{fig:lookaheadsplit}
\end{figure}

\subsection{NFA Simulation}
\index{NFA!Simulation}

The Regex engine in Java does not convert to a Thompson-DFA; it uses a backtracking algorithm
to find out if a regular expression matches a string.  This leads to pathological cases with
exponential runtime increase, particularly when the regular expression contains a large number
of Kleene stars.

In those situations, it may be helpful to construct your own mini-regexp interpreter by building
and simulating an NFA (nondeterministic finite automaton).

Example problem is \href{http://ncpc.idi.ntnu.no/ncpc2011/ncpc2011problems.pdf}{NCPC 2011/E}
where the input are globs such as \texttt{*a*a*a*a} that should be matched against filenames.

\inputminted{python}{code/ls.py}

\section{Parsing}
\subsection{Recursive Descent}

 
\chapter{Mathematics}

\section{Combinatorics}

A number of problems require basic knowledge of combinatorics.  
We have seen two characteristics of these problems:
\begin{itemize}
\item They are designed such that brute-force approaches that include the enumeration of
    permutations or combinations will time out.  A telltale sign is if the problem allows
    ranges for its input parameters that exceed 32 bits ($2^{32} \sim 4 x 10^9$) or allow
    for up to $10^{18}$.  Reminder: $2^{63} \sim 9 x 10^{18}$
\item Because the input sizes are such that brute-force enumeration is ruled out, there
    is a risk of integer overflow when computing binomial cofficient and factorials.
\end{itemize}

\index{Permutations}
\textbf{Permutations of length $k$ of $n$ elements, allowing for repetitions}: 
\[
    n^k
\]
Important special case is $n=2$ - number of permutations
of length $k$ is equal to number of ways in which to form $k$ bits ($2^k$).

\textbf{Permutations of $n$ elements, no repetitions}: 
\[
    n! = n (n-1) (n-2) \ldots 1
\]

\inputjava{bigintegerfactorial}

\index{Factorial}
\textbf{Number of unique permutations when elements occur multiple times.}  Assume $a_i$ may be repeated $k_i$ times. 
    Let $L = \sum_{i = 1}^{n} k_i$ the sum of their frequencies.

\[
    \frac{L!}{k_1! k_2! \ldots ... k_n!}
\]
Example: unique permutations of (aabb) is $L = 4$, so $\frac{4!}{2! 2!} = 6$.  The permutations are
(aabb), (abab), (abba), (baab), (baba), (bbaa).

\index{Combinations}
\index{Binomial Coefficient}
\textbf{Combinations of $k$ out of $n$ elements, no repetition}:
\[
    \binom{n}{k} = \frac{n!}{k! (n-k)!}
\]

Aside: C++ is a valid language at the ICPC, but unlike Java, C++ does not have a standard
library for arbitrary-size integers that is accessible during the contest.  This has led to problems in which
the problem states that the result fits into a 64-bit long, but where the straightforward
application of the formula for $n!$ or $\binom{n}{k}$ would lead to integer overflow.

In those cases, it is recommended to perform the arithmetic in BigInteger and then convert
down if/when needed, as in the example below:

\inputjava{bigintegerbinom}

Note that Java does not have an unsigned 64-bit type, so above code will fail for 
coefficients that are $2^{63} \leq \binom{n}{k} < 2^{64}$.  
For the same reason, however, judge data usually avoids that range.
Code above is from solution to 2010/D Bit~\ref{sec:2010-d-bit}

\textbf{Combinations of $k$ out of $n$ elements, can repeat any element any number of times}:
\[
    \binom{n + k - 1}{k}
\]

 
\chapter{Geometry}

\section{Basics}

A determinant of a $2x2$ matrix is defined as
\[
    \left\vert
        \begin{array}{cc}
            a & b \\
            c & d \\
        \end{array}
    \right\vert
    = a d - b c
\]

\section{java.awt.geom}

The \texttt{java.awt.geom} and \texttt{java.awt} packages have, albeit limited, facilities
for geometric problems.  There are classes to represent shapes - see
\href{http://docs.oracle.com/javase/6/docs/api/java/awt/Shape.html}{java.awt.Shape}, including
lines, ellipses, rectangles and some curves.

\begin{itemize}
\item "is contained in".  java.awt.geom.Shape provide a contains() method to test if a point
    is contained in a shape.  Contains() returns true if the point is in the interior, and false
    if the point is outside the shape. However, it \textbf{may return true or false if the point is 
    on the shape boundary.} 

\begin{figure}
    \centering
    \includegraphics[height=2in]{outcode.pdf}
    \caption{Outcode - note that points that lie on any of the sidelines are considered inside.}
    \label{fig:outcode}
\end{figure}

\item "outcode". Outcodes were invented by Cohen-Sutherland; they are used for clipping
    in computer graphics.  java.awt.geom.Rectangle2D provides an outcode() method.
    The result is 0 if a point is inside or on any sideline of the rectangle; otherwise 
    the result is a
    combination of bits that represent where the point lies in relationship to the
    rectangle, as shown in Figure~\ref{fig:outcode}.
    For clipping of lines, the outcodes of the start and end point are
    computed, which then allows a quick identification of whether the line is inside,
    must be clipped, may be ignored, or needs further investigation.
    A useful property of outcode() is that it can substitute as a replacement for
    contains() in case where a point may lie on an edge but should be considered
    inside. 

\item "intersects."  Tests if a shape intersects with a rectangle.
    Can also test if two lines or line segments intersect, but cannot find the point of
    intersection.

\item "is point on line segment." Implements this as Line2D.ptSegDistSq(Point2D) $<$ 1e-9.

\end{itemize}

\section{Coordinate Geometry}

\subsection{Line/Line Intersection}
\label{sec:lineintersection}
\index{Line/Line Intersections}

\begin{figure}
    \centering
    % Wikipedia Public Domain image
    \includegraphics[height=1.5in]{Line-Line_Intersection.png}
    \caption{Line line intersection}
    \label{fig:linelineintersect}
\end{figure}

\[
\begin{array}{ll}
P_x = \frac{\begin{vmatrix} 
                \begin{vmatrix} x_1 & y_1\\
                                x_2 & y_2
                \end{vmatrix} & 
                \begin{vmatrix} x_1 & 1\\
                                x_2 & 1
                \end{vmatrix} \\\\ 
                \begin{vmatrix} x_3 & y_3\\
                                x_4 & y_4
                \end{vmatrix} & 
                \begin{vmatrix} x_3 & 1\\
                                x_4 & 1
                \end{vmatrix} 
            \end{vmatrix} }{
            \begin{vmatrix} 
                \begin{vmatrix} x_1 & 1\\
                               x_2 & 1
                \end{vmatrix} &  
                \begin{vmatrix} y_1 & 1\\
                                y_2 & 1
             \end{vmatrix} \\\\ 
             \begin{vmatrix} x_3 & 1\\
                             x_4 & 1
             \end{vmatrix} & 
             \begin{vmatrix} y_3 & 1\\
                             y_4 & 1 
             \end{vmatrix} 
            \end{vmatrix}}

&

P_y = \frac{\begin{vmatrix} 
                \begin{vmatrix} x_1 & y_1\\
                                x_2 & y_2
                \end{vmatrix} &  
                \begin{vmatrix} y_1 & 1
                              \\y_2 & 1
                \end{vmatrix} \\\\ 
                \begin{vmatrix} x_3 & y_3\\
                                x_4 & y_4
                \end{vmatrix} & 
                \begin{vmatrix} y_3 & 1\\
                                y_4 & 1
                \end{vmatrix} 
            \end{vmatrix} }
           {\begin{vmatrix} 
               \begin{vmatrix} x_1 & 1\\
                               x_2 & 1
               \end{vmatrix} &
               \begin{vmatrix} y_1 & 1\\
                               y_2 & 1
            \end{vmatrix} \\\\ 
            \begin{vmatrix} x_3 & 1\\
                            x_4 & 1
            \end{vmatrix} & 
            \begin{vmatrix} y_3 & 1\\
                            y_4 & 1
            \end{vmatrix} 
     \end{vmatrix}}\,\!
\\

\end{array}
\]

The determinants can be written out as:
\begin{align*}
    (P_x, P_y)= \bigg(&\frac{(x_1 y_2-y_1 x_2)(x_3-x_4)-(x_1-x_2)(x_3 y_4-y_3 x_4)}{(x_1-x_2)(y_3-y_4)-(y_1-y_2)(x_3-x_4)}, \\
                      &\frac{(x_1 y_2-y_1 x_2)(y_3-y_4)-(y_1-y_2)(x_3 y_4-y_3 x_4)}{(x_1-x_2)(y_3-y_4)-(y_1-y_2)(x_3-x_4)}\bigg)
\end{align*}

Source: \href{http://en.wikipedia.org/wiki/Line-line_intersection}{http://en.wikipedia.org/wiki/Line-line\_intersection}.

\paragraph{Notes}
\begin{itemize}
\item Does not handle parallel or coincident lines:
    Denominator will be zero:
    \[
        (x_1 - x_2) (y_3 - y_4) - (y_1 - y_2) (x_3 - x_4) = 0
    \]
\item Does not handle if lines are each others' normal (i.e., at a right angle).
    If line is horizontal ($y_1 = y_2$ or $y_3 = y_4$), and the other vertical ($x_1 = x_2$ or $x_3 = x_4$) 
    denominator will also be a 0 determinant, but the lines will intersect.  
    Handle as special case if problem allows it.

\item Intersection point may be outside the given segments.

\item If you only need to know if two lines intersect, but not where, use java.awt.geom.Line2D.intersects.
\end{itemize}

\paragraph{Code}

This code is from a solution to 2011/F (Section~\ref{sec:2011-f-lineofsight}) where the 
parallel and rectangular cases do not occur. (TBD: provide complete implementation.)

\inputminted[fontsize=\footnotesize,linenos=true]{java}{code/lineintersection.java}

%
%
%

\subsection{Area of a Polygon}
\label{sec:areapolygon}
\index{Polygon!Area}
\index{Area!Polygon}
\index{Polygon}

The signed area of a planar non-self-intersecting polygon with vertices $(x_1, y_1), \dots, (x_n, y_n)$ is
\[
    A = \frac{1}{2} \left(
        \left\vert
        \begin{array}{cc}
            x_1 & x_2 \\
            y_1 & y_2 \\
        \end{array}
        \right\vert
        +
        \left\vert
        \begin{array}{cc}
            x_2 & x_3 \\
            y_2 & y_3 \\
        \end{array}
        \right\vert
        + \ldots +
        \left\vert
        \begin{array}{cc}
            x_n & x_1 \\
            y_n & y_1 \\
        \end{array}
        \right\vert
        \right)
\]

Figure~\ref{fig:polygonareadeterminant} shows how to multiply this out
\[
    A = \frac{1}{2} \left(
        x_1 y_2 - x_2 y_1
      + x_2 y_3 - x_3 y_2
      + \ldots +
      + x_{n-1} y_n - x_n y_{n-1}
      + x_{n} y_1 - x_1 y_n
      \right)
\]

(Source: Mathworld~\cite{mathworldpolygonarea})

\begin{figure}
    \centering
    % Wikipedia Public Domain image
    \includegraphics[height=1.5in]{PolygonArea_1000.png}
    \caption{Line line intersection}
    \label{fig:polygonareadeterminant}
\end{figure}

\paragraph{Notes}
\begin{itemize}
\item Works for any simple polygon (concave or convex)
\item Does not work for complex polygons (when any edges intersect)
\item Points \textbf{must be ordered} if polygon has more than 3 vertices, or output is junk.
\item A is positive if points are in counterclockwise order, negative if points are in clockwise order.
    See the use of Math.abs() in code below.
\item Triangle and any Quadrilateral are, of course, just special cases.
    For triangles, order does not matter.
\end{itemize}

\paragraph{Code}
\inputminted[fontsize=\footnotesize,linenos=true]{java}{code/polygonarea.java}

Special case of a triangle:

\inputminted[fontsize=\footnotesize,linenos=true]{java}{code/triangleareacoord.java}

%%%%%%%%%%%%%%%%%%%%%%%%%%%%%%%%%%%%%%%%%%%%%%%%%%%%%%%%%%%%%%%%%%%%%%%%%%%%%%%%%%%%%%%%%%%
%
%
\subsection{Convex Hull}
\label{sec:convexhull}
\index{Polygon!Convex Hull}
\index{Convex Hull}

A frequent favorite is the computation of the convex hull of a set of points in the plane,
defined as the smallest convex polygon that encloses all points.

\begin{figure}
    \centering
    % Wikipedia Public Domain image
    \includegraphics[height=2in]{500px-ConvexHull.png}
    \caption{Convex Hull - Rubber Band analogy. The convex hull is the convex polygon created 
        when spanning a rubber band around a set of points.}
    \label{fig:convexhull}
\end{figure}

 
\chapter{Gotchas}

Common mistakes and idiosyncrasies observed in the judge input and specification of
various problems posed at competitions.

\begin{enumerate}
\item \textbf{Judge input not terminated as required}. Typically, the problem states that
    there's some way to identify the end of input without having to rely on EOF.
    We've observed judge input, however, where EOF terminated the input.
    You should try to write your input loop such that your solution works whether
    the input is terminated by EOF or by the specified end-of-input delimiter.
    This strategy may allow you to submit a correct solution even before the mistake
    is discovered (and may even lead to a delay in when it's discovered that would benefit
    your team.)
    Seen in 2006/E Marbles~\ref{sec:2006-e-marbles}.

\item \textbf{Insignificant Trailing Spaces}.
    In problems that state "there is one word per line" we have observed trailing spaces
    which must be trimmed.   Advice: always use String.trim(), unless the spaces are
    significant, which we have not seen anywhere.
    Seen in 2007/D Witness~\ref{sec:2007-d-witness}.

\item \textbf{Insignificant Leading Spaces}.
    In some problems, (insignificant) leading spaces may occur.  The catch here is that 
    naive splitting without trimming may produce an empty string in the first position.
    See bsh output below:
    \begin{Verbatim}
    % System.out.println(Arrays.toString("  word1  word2  ".split("\\s+")));
    [, word1, word2]
    \end{Verbatim}
 
    Seen in 2011/B Raggedy~\ref{sec:2011-b-raggedy}.

\item \textbf{Significant Leading and Trailing Spaces}.
    Check if the problem allows for significant leading and trailing spaces, such as when
    reading in grids or mazes or specified width/height.  Don't accidentally trim(), but don't
    expect the trailing spaces, either.  Consider allocating and filling an array of desired
    length and using System.arraycopy to copy the input.

    Seen in 2011/H Road Rally~\ref{sec:2011-h-rally}.

\end{enumerate}

 
\chapter{Mid-Atlantic Problem Sets}

This chapter contains some notes about the problems occurring in the Mid-Atlantic
problem set.  We focus on this corpus in particular because there are recurring
themes since the problems have been created by the same person (or team) for
multiple years.

\section{2005}

\href{(Problem PDF 2005)}{http://midatl.radford.edu/docs/pastProblems/05contest/MidAtlantic2005.pdf}

\subsection{C Extrusion}
\label{sec:2005-c-extrusion}

Straightforward application of polygon area formula, see Section~\ref{sec:areapolygon}.

\section{2006}
\href{(Problem PDF 2006)}{http://midatl.radford.edu/docs/pastProblems/06contest/MidAtlantic2006.pdf}

\subsection{E Marbles}
\label{sec:2006-e-marbles}
A simple state space exploration problem solvable with straightforward BFS exploration.
Catch: judge input data missed the "0 0 0" line.

\section{2007}
\href{(Problem PDF 2007)}{http://midatl.radford.edu/docs/pastProblems/07contest/MidAtlantic2007.pdf}

\subsection{B Mobiles Alabama}
\label{sec:2007-b-mobile}

Lexical analysis benefits from zero-width lookaround~\ref{sec:lookaroundsplitting}, although
simpler solutions (replacing '(' and ')' with ' ( ' and ' ) ' before splitting on whitespace
may work, too.
Recursive descent parsing should be used to analyse the syntactical structure of the input.

\subsection{D Witness Redaction}
\label{sec:2007-d-witness}
This problem can be solved with regular expressions and zero-width lookaround splitting.
See Section~\ref{sec:lookaroundsplitting}.

\section{2008}
\href{(Problem PDF 2008)}{http://midatl.radford.edu/docs/pastProblems/08contest/MidAtlantic2008.pdf}

\subsection{G Stems Sell}
\label{sec:2008-g-stems}

Can be solved with regular expressions.

\href{Judge data}{http://midatl.radford.edu/docs/pastProblems/08contest/JudgingData/G-stems/}
appears broken, even on ICPC site.

\section{2011}

\href{(Problem PDF 2011)}{http://midatl.radford.edu/docs/pastProblems/11contest/MidAtlantic2011.pdf}

\subsection{B Raggedy, Raggedy}
\label{sec:2011-b-raggedy}

This problem can be solved using dynamic programming.
Note that there may be leading spaces on some input lines.

\subsection{F Line of Sight}
\label{sec:2011-f-lineofsight}

Straightforward application of area of polygon~\ref{sec:areapolygon} and line intersection~\ref{sec:lineintersection}.
Note that parameters of the problem even exclude corner cases for line intersection (e.g. parallel lines, right angles).

 

\backmatter 
%\include{glossary} 
%\include{notat} 

\bibliographystyle{plain} %The style you want to use for references. 
\bibliography{references} %The files containing all the articles and books you ever referenced. 

\printindex

\end{document}

