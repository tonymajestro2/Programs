\chapter{Mathematics}

\section{Combinatorics}

A number of problems require basic knowledge of combinatorics.  
We have seen two characteristics of these problems:
\begin{itemize}
\item They are designed such that brute-force approaches that include the enumeration of
    permutations or combinations will time out.  A telltale sign is if the problem allows
    ranges for its input parameters that exceed 32 bits ($2^{32} ~ 4 x 10^9$) or allow
    for up to $10^{18}$.  Reminder: $2^{63} ~ 9 x 10^{18}$
\item Because the input sizes are such that brute-force enumeration is ruled out, there
    is a risk of integer overflow when computing binomial cofficient and factorials.
\end{itemize}

\textbf{Permutations of length $k$ of $n$ elements, allowing for repetitions}: 
\[
    n^k
\]
Important special case is $n=2$ - number of permutations
of length $k$ is equal to number of ways in which to form $k$ bits ($2^k$).

\textbf{Permutations of $n$ elements, no repetitions}: 
\[
    n! = n (n-1) (n-2) \ldots 1
\]

\inputjava{bigintegerfactorial}

\textbf{Number of unique permutations when elements occur multiple times.}  Assume $a_i$ may be repeated $k_i$ times. 
    Let $L = \sum_{i = 1}^{n} k_i$ the sum of their frequencies.

\[
    \frac{L!}{k_1! k_2! \ldots ... k_n!}
\]
Example: unique permutations of (aabb) is $L = 4$, so $\frac{4!}{2! 2!} = 6$.  The permutations are
(aabb), (abab), (abba), (baab), (baba), (bbaa).

\textbf{Combinations of $k$ out of $n$ elements, no repetition}:
\[
    \binom{n}{k} = \frac{n!}{k! (n-k)!}
\]

Aside: C++ is a valid language at the ICPC, but unlike Java, C++ does not have a standard
library for arbitrary-size integers that is accessible during the contest.  This has led to problems in which
the problem states that the result fits into a 64-bit long, but where the straightforward
application of the formula for $n!$ or $\binom{n}{k}$ would lead to integer overflow.

In those cases, it is recommended to perform the arithmetic in BigInteger and then convert
down if/when needed, as in the example below:

\inputjava{bigintegerbinom}

Note that Java does not have an unsigned 64-bit type, so above code will fail for 
coefficients that are $2^{63} <= \binom{n}{k} < 2^{64}$.  
For the same reason, however, judge data usually avoids that range.
Code above is from solution to 2010/D Bit~\ref{sec:2010-d-bit}

\textbf{Combinations of $k$ out of $n$ elements, can repeat any element any number of times}:
\[
    \binom{n + k - 1}{k}
\]

