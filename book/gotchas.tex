\chapter{Gotchas}

\label{sec:gotchas}

Common mistakes and idiosyncrasies observed in the judge input and specification of
various problems posed at competitions.

\begin{enumerate}
\item \textbf{Judge input not terminated as required}. Typically, the problem states that
    there's some way to identify the end of input without having to rely on EOF.
    We've observed judge input, however, where EOF terminated the input.
    You should try to write your input loop such that your solution works whether
    the input is terminated by EOF or by the specified end-of-input delimiter.
    This strategy may allow you to submit a correct solution even before the mistake
    is discovered (and may even lead to a delay in when it's discovered that would benefit
    your team.)
    Seen in 2006/E Marbles~\ref{sec:2006-e-marbles}.

\item \textbf{Insignificant Trailing Spaces}.
    In problems that state "there is one word per line" we have observed trailing spaces
    which must be trimmed.   Advice: always use String.trim(), unless the spaces are
    significant, which we have not seen anywhere.
    Seen in 2007/D Witness~\ref{sec:2007-d-witness}.

\item \textbf{Insignificant Leading Spaces}.
    In some problems, (insignificant) leading spaces may occur.  The catch here is that 
    naive splitting without trimming may produce an empty string in the first position.
    See bsh output below:
    \begin{Verbatim}
    % System.out.println(Arrays.toString("  word1  word2  ".split("\\s+")));
    [, word1, word2]
    \end{Verbatim}
 
    Seen in 2011/B Raggedy~\ref{sec:2011-b-raggedy}.

\item \textbf{Significant Leading and Trailing Spaces}.
    Check if the problem allows for significant leading and trailing spaces, such as when
    reading in grids or mazes or specified width/height.  Don't accidentally trim(), but don't
    expect the trailing spaces, either.  Consider allocating and filling an array of desired
    length and using System.arraycopy to copy the input.

    Seen in 2011/H Road Rally~\ref{sec:2011-h-rally}.

\item \textbf{Forgetting the ``input is solution'' case.}
    In some problems, it may be that the input is already the desired solution, though the
    sample data usually does not cover the case.
    Make sure your code handles this.  For instance, in a BFS, it's easy to miss if you only
    test successor states for finality.
    Seen in 2006/E Marbles~\ref{sec:2006-e-marbles}.

\item \textbf{Forgetting the "n = 1" case.}
    Make sure that you don't rely on there being at least 2 items in sorting and other
    problems, unless the problem states that.  Notably, if $n = 1$, a comparator
    function might not be invoked.
    The code below attempts to be clever and save a loop over \code{sl} by setting the 
    \code{wLoss} map entry inside the comparator.

    \inputminted[fontsize=\footnotesize,linenos=true]{java}{code/sortn1bug.java}

    Seen in 2011/E Losers are Winners~\ref{sec:2011-e-losers}.

\end{enumerate}

