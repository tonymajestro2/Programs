\chapter{Searching}
\def\astar{$A^*$}

\section{Backtracking}
\label{sec:backtracking}
\index{Backtracking}

Bitner~\cite{Bitner:1975} first wrote up the underlying idea of backtracking in 1975.
While some ideas of that paper (notably, the use of macros) reflect specific implementation
ideas of that time, the rest is still relevant.

A canonical example of simple backtracking is to solve the N-Queens problem - arrange N queens
on a chessboard of size NxN such that no two can attack each other.  In this case, each
row must contain exactly one queen; the backtracking algorithm places queens on each square
in rows 1, 2, ..., N until the next queen either cannot be placed or a solution is found.

\inputjava{NQueens}

\paragraph{General Structure.}
The general structure of a backtracking function is
\begin{verbatim}
backtrack(<arguments that reflect partial solution> args)
{
    if (args say solution is found) {
        // process solution
        return <a solution was found>;
    }

    S = sort possible next steps by likelihood of finding a solution

    for (Steps s in S) {
        augment partial solution with s
        if (augmented solution is still feasible)
            backtrack(args');
        remove s from partial solution (undo)
    }
    return <no solution found>
}
\end{verbatim}
Alternatively, it is often convenient to clone the partial solution before augmenting it
with 's', which makes the undo step unnecessary.

\paragraph{Notes}
\begin{itemize}
\item Continue here.
\end{itemize}
 

\section{State Space Exploration}

In these problems, an initial state is to be transformed through a series of valid
moves into one goal state, or possibly one of multiple possible goal states. 
Examples include block puzzles or single-player games.
These problems have the following characteristics:
\begin{itemize}
\item An optimal solution required: we want the minimal number of moves from the
initial state to goal state, rather than just any.

\item No obvious strategy.  Given a state, there's no obvious way to choose which
move to make to get closer to the goal.  In fact, it's usually even difficult to
tell how far we might be away from the goal state.  Any easy-to-find lower bounds
for this distance to the goal might be far too optimistic.  
See also \astar{} in Section~\ref{sec:astar}.

\end{itemize}

\subsection{Breadth-First Search (BFS)}
\label{sec:bfs}
There are numerous ways of writing a BFS loop.  There are even more ways of
getting it wrong.
Here is one possible way:

\inputminted[fontsize=\footnotesize,linenos=true]{java}{code/bfsloop.java}

\paragraph{Notes.}  
\begin{itemize}
\item Adding the final state could be avoided by calling output() if a 
final state is found, potentially saving the unnecessary expansion of states
if the optimal goal state is already in the queue - however, then the case where 
the initial state is final must be handled separately.  
Seen in 2006/E Marbles~\ref{sec:2006-e-marbles} where the judge data contains
the case that the initial state is final.

\item The state class should use ducktyping and provide the necessary
methods - isfinal, successors.  In addition, the state class must implement
an object equivalence relationship (equals, hashCode) as described in 
Section~\ref{sec:objequivalence}.

\item The example keeps track of both the path (via 'pred') and distance (via 'dist').
    If the problem asks for only one of the two, the other can be omitted. In that
    case, the insertion of a successor state should be guarded by the one remaining.
    Note that both 'pred' and 'dist' have the invariant when encountering a state 
    multiple times, only the first encountered state is kept in the map.

\end{itemize}

\paragraph{When Not To Use BFS.}

There are some problems that may appear to be solvable using BFS, but are in fact 
dynamic programming problems.  These are generally problems in which there are
many transitions from states farther in the problem space to states that have
been discovered much earlier.
An example is 2007/C/Out of Sight.~\ref{sec:2007-c-sight}.

\subsection{\astar}
\label{sec:astar}

\astar~\cite{Pearl:1984} is a classic algorithm to improve upon breadth-first search when an
estimator function for states that estimates their distance to the goal state is known.
Though an AI algorithm, it can yield guaranteed optimal solution if the estimator is chosen
carefully.

The idea is simple.  In BFS, we're examining nodes based on their distance from the start state. 
So we'll look at, and expand, all states that are 6 moves away from the start *before* we 
look at any state that's 7 moves from the start - simply because of the FIFO discipline in the queue.

In \astar, instead of putting all to-be-explored states in a FIFO queue,
we sort them by their "goodness," which is defined as a
lower bound of how many moves a solution is away from the state.
For instance, if a state A is 6 moves away from the start, and
it is known that it is at least 10 moves away from the goal,
it would have a goodness of 16 (6 + 10).  On the other hand, if a
move B is 8 moves away from the start, and at least 4 moves away
from the solution, its goodness is 12 and it's explored first.
Note that we are not estimating how far the state is from
the solution. We're simply constructing a function that says: this
state is *at least* this far from a solution. It may be farther -
our decision to explore state B may be wrong and the closest solution
state may result from the exploration of A.

The challenge lies in how to say - quickly - that a state is
``at least this many moves away from the goal.''  In general, that's
tough since if we knew how many states a state is away from the goal,
it would be much easier to solve the problem. However, there are some lower
bounds that can be found - for instance, in a puzzle, we could count how
many pieces are not in their final position, knowing that it'll take at
least this many moves to get them there (assuming it's a puzzle like
the traditional 15-piece puzzle where all pieces need to be sorted).
Another possible function is the sum of the Manhattan
distances between each piece's current and final position.  
Again, it is clear that it will take at least this many moves.

In competition problems you are practically never asked to find an 'almost'
perfect heuristic solution - they ask
you to compute the correct, and in search problems like these, the
shortest solution.  \astar{} will provide you with the optimal
solution if the estimation function does not overestimate  - if it
did, it might lead to the nearest goal state being placed behind a
farther goal state in the queue.  That farther state might then be
discovered first, and mistaken for the shortest solution.

It's not a problem that the estimation function underestimates - in fact,
it'll generally do that - this just means that \astar might waste some
time exploring states it thinks are promising, but which do not actually
lead to a solution.

Converting a BFS to A-Star in Java is really simple.  Replace the entry
of the BFS described in Section~\ref{sec:bfs} such that the BFS work
queue uses a PriorityQueue like so:

\inputminted[fontsize=\footnotesize,linenos=true]{java}{code/astar.java}

Note how making dist final allows it to be used in the comparator.
The code exploits that both PriorityQueue and ArrayDeque implement the
Queue interface.

\paragraph{Notes}  
\begin{itemize}
\item
\astar will increase the cost of inserting and removing a state from the queue
substantially - in a heap-based priority queue, 'offer()' and 'poll()' take O(log n) whereas 
they are constant time O(1) operations on a Deque.  On the other hand, especially if the
estimator function is too optimistic, the state space may only be marginally reduced.
Recommendation: use A-Star only if BFS times out and/or if a good heuristic
can be found.

\end{itemize}
