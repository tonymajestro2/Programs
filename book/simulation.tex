\chapter{Simulation}

While many simulation problems can be solved ad-hoc, some require, or may benefit
from, a more principled approach.

\section{Monte Carlo Methods}

In some cases, random sampling might provide a solution.
For instance, the code below computes $\pi$ by examining whether
randomly produced $(x, y)$ samples are inside or outside the unit 
circle in the northeast quadrant.

\inputminted[fontsize=\footnotesize,linenos=true]{java}{code/Pi.java}

\paragraph{Precision} is generally not very good, mainly due to the  
statistical properties of the underlying pseudo-random number generator (PRNG).
Increasing the number of samples will yield diminishing returns quickly.
As a rule of thumb, don't expect more than 4 significant digits.
For example, 2010/Cells~\ref{sec:2010-c-cells} asks for a percentage with 2 digits after the
period, for a total of 4 significant digits, making this approach feasible.

\section{Discrete Event Simulation}

Discrete Event Simulation is widely used to simulate physical phenomena, especially
those that involve periodic and/or random events, as well as when events may or may
not occur, depending on the state of the simulation.

The key idea is to maintain a timeline as a queue of future events, sorted
by their timestamp.  Simulation time, aka virtual time, advances in jumps when
the next upcoming event is pulled off the queue.  Time jumps over period in which
no events are scheduled (unlike a continuous simulation in which time increases
in small, but constant, steps).

Event handlers may query the current time and they may schedule future events.
Discrete Event simulations contain a typical core: a way to represent events, 
a queue to sort them, and an event loop that processes the event queue until
either the maximum simulation time is reached or a solution to the problem is found.
Recommended implementation strategy is to use immutable event objects that are
inserted when scheduled and discarded after dispatch.

Example: \href{http://uva.onlinejudge.org/index.php?option=com_onlinejudge&Itemid=8&category=3&page=show_problem&problem=97}{UVA 00161}
is an example of a problem which can be solved with DES.  Note, however, that the small simulation time
frame (18,000 seconds) also allows a continuous simulation approach, simply simulating every second,
rather than only those points in time when a traffic light changes.
This is a very simple example, with only 3 event types, all implemented in one
class LightChange.

\inputminted[fontsize=\footnotesize,linenos=true]{java}{code/UVA00161TrafficLights.java}

