\section{Backtracking}
\label{sec:backtracking}
\index{Backtracking}

Bitner~\cite{Bitner:1975} first wrote up the underlying idea of backtracking in 1975.
While some ideas of that paper (notably, the use of macros) reflect specific implementation
ideas of that time, the rest is still relevant.

A canonical example of simple backtracking is to solve the N-Queens problem - arrange N queens
on a chessboard of size NxN such that no two can attack each other.  In this case, each
row must contain exactly one queen; the backtracking algorithm places queens on each square
in rows 1, 2, ..., N until the next queen either cannot be placed or a solution is found.

\inputjava{NQueens}

\paragraph{General Structure.}
The general structure of a backtracking function is
\begin{verbatim}
backtrack(<arguments that reflect partial solution> args)
{
    if (args say solution is found) {
        // process solution
        return <a solution was found>;
    }

    S = sort possible next steps by likelihood of finding a solution

    for (Steps s in S) {
        augment partial solution with s
        if (augmented solution is still feasible)
            backtrack(args');
        remove s from partial solution (undo)
    }
    return <no solution found>
}
\end{verbatim}
Alternatively, it is often convenient to clone the partial solution before augmenting it
with 's', which makes the undo step unnecessary.

\paragraph{Notes}
\begin{itemize}
\item Continue here.
\end{itemize}
